\documentclass[stdletter,letterpaper,addrfromright,orderfromdateto,dateleft,11pt,noaddrto,noaddrfrom,sigleft]{newlfm}
\topmarginskip{-0.2in}
\bottommarginskip{-1.5in}
\leftmarginsize{1in}
\rightmarginsize{1.25in}
\sigskipbefore{0.4in}
\sigskipafter{0in}
\noLines
\nolines
\noHeadline
\noheadline
\signature{Andrew Rambaut and Trevor Bedford\\ \textsl{and on behalf of Mercedes Pascual}}

\namefrom{}
\addrfrom{Dept.\ of Ecology and Evolutionary Biology \\ University of Michigan \\ 830 North University Avenue \\ Ann Arbor, MI 48109}
\phonefrom{(617) 285-2542}
\faxfrom{(734) 763-0544}
\emailfrom{bedfordt@umich.edu}

\nameto{BioMed Central Ltd}
\addrto{236 Gray's Inn Road \\ London, WC1X 8HB}

\greetto{Dear Editor,}
\closeline{Sincerely,}

% comments
\usepackage{color} 
\usepackage{ulem}
\definecolor{purple}{rgb}{0.459,0.109,0.538}
\def\tb#1#2{\sout{#1} \textcolor{purple}{#2}} 
\def\tbc#1{\textcolor{purple}{[#1]}}
\definecolor{blue}{rgb}{0.324,0.609,0.708}
\def\sc#1#2{\sout{#1} \textcolor{blue}{#2}} 
\def\scc#1{\textcolor{blue}{[#1]}}
\definecolor{green}{rgb}{0.513,0.73,0.442} 
\def\mp#1#2{\sout{#1} \textcolor{green}{#2}} 
\def\mpc#1{\textcolor{green}{[#1]}}
 
\begin{document}

\begin{newlfm}

Please find, attached, our manuscript entitled ``Canalization of the evolutionary trajectory of the human influenza virus,'' which we would be grateful for your consideration for publication in \textsl{Nature}.  We believe that \textsl{Nature} is an appropriate venue for this work, as it addresses long-standing controversies about the evolution and the epidemiology of the human influenza virus, with important implications for public health.

In many ways, the evolution and epidemiology of seasonal human influenza has remained mysterious.  It is clear that evolution occurs rapidly allowing the virus to escape from the build-up of human immunity.  Antigenically novel strains of influenza continually arise and replace more primitive `spent' strains.  Observed patterns of antigenic drift show that antigenic evolution occurs in a predominately \textsl{linear} fashion.  Rather than diversification through time, influenza shows serial replacement of antigenic types.  This pattern has remained puzzling from an epidemiological standpoint, and previous work has sought explanations in mechanisms such as short-lived strain-transcending immunity and limited antigenic repertoires.  Here, we find that a simple geometric model of antigenic space simultaneously explains observed genetic and antigenic data.  We harness a large-scale individual-based simulation to show that evolution under such a model appears \textsl{canalized}; rather than moving in multiple antigenic directions, the virus population is shunted down a narrow path, constrained by the actions of natural selection.  Thus, the myopic perspective of natural selection sacrifices long-term gains in antigenic diversification in exchange for short-term advantages.

This work is the first to simultaneously explain detailed epidemiological, genealogical, antigenic and spatial patterns in the human influenza virus.  In uniting data sources, we provide the foundations for a \textsl{predictive} modeling effort.  We show that on shorter timescales of 1--2 years, the evolution of the influenza virus appears exceptionally repeatable.  Building on this work, it should be possible to better assess the likelihood that an observed antigenic variant will take-off, and warrant a matching vaccine update.  Additionally, our model makes concrete predictions about the details of the antigenic space influenza evolves within.  Furthermore, our novel computational approach allows us to directly track the genealogical history of the simulated epidemic, rather than relying on the substantially slower and less accurate phylogenetic methods previously in widespread use.  We are confident that the results and methodology described in the manuscript will shape future study of the evolution and epidemiology of influenza, but also impact the general study of the phylodynamics of evolving human pathogens.

% "When submitting new or revised manuscripts, authors should state in a cover letter to the editor their rough estimate of the length of their paper in terms of number of pages of Nature."

The manuscript totals approximately four printed pages of \textsl{Nature}, comprising 2189 words of main text and 4 figures.

\end{newlfm}
\end{document}  