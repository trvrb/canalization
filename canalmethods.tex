\section*{Methods}

\subsection*{Transmission model}

To characterize the joint epidemiological, genealogical, antigenic and spatial patterns of influenza, we implemented a large-scale individual-based model.  This model consists of daily time steps, in which the states of hosts and viruses are updated.  Hosts may be born, may die, may contact other hosts allowing viral transmission, or may recover from infection.  Viruses may mutate in antigenic phenotype.  Each simulation ran for 40 years of model time.  

Hosts in this model are divided between three regions: North, South and Tropics.  There are 30 million hosts within each of the three regions, giving $N = 9 \times 10^{7}$ hosts.  Host population size remains fixed at this number, but vital dynamics cause births and deaths of hosts at a rate of $1 / 30$ years $= 9.1 \times 10^{-5}$ per host per day.  Within each region, transmission proceeds through mass-action with contacts between hosts occurring at an average rate of $\beta = 0.36$ per host per day.  Regional transmission rates in temperate regions vary according to sinusoidal seasonal forcing with amplitude $\epsilon = 0.15$ and opposite phase in the North and in the South.  Transmission rate does not vary over time in the Tropics.  Transmission between region $i$ and region $j$ occurs at rate $m\,\beta_i$, where $m=0.001$ and is the same between each pair of regions and $\beta_i$ is the within-region contact rate.   Hosts recover from infection at rate $\nu = 0.2$ per host per day, giving $R_0$ in a naive host population of 1.8.  There is no super-infection in the model.

Each virus possesses an antigenic phenotype, represented as a location in Euclidean space.  Here, we primarily use a two-dimensional antigenic location.  After recovery, a host `remembers' the phenotype of its infecting virus as part of its immune history.  When a contact event occurs and a virus attempts to infect a host, the Euclidean distance from infecting phenotype $\phi_v$ is calculated to each of the phenotypes in the host immune history $\phi_{h_1}, \dots, \phi_{h_n}$.  Here, one unit of antigenic distance is designed to correspond to a twofold dilution of antiserum in a hemagglutination inhibition (HI) assay \cite{Smith04}. The probability that infection occurs after exposure is proportional to the distance $d$ to the closest phenotype in the host immune history.  Risk of infection follows the form $\rho = \textrm{max}\{d\,s,1\}$, where $s=0.07$.  Cross-immunity $\sigma$ equals $1-\rho$.  The initial host population began with enough immunity to slow down the initial virus upswing and place the dynamics closer to their equilibrium state; initial $R$ was 1.28.

The initial virus population consisted of 10 infections each with the identical antigenic phenotype of $\{0,0\}$.  Over time viruses evolve in antigenic phenotype.  Each day there is a chance $\mu = 10^{-4}$ that an infection mutates to a new phenotype.  This mutation rate represents a phenotypic rate, rather than genetic mutation rate, and can be thought of as arising from multiple genetic sources.  When a mutation occurs, the virus's phenotype is moved in a completely random direction $\sim \textrm{Uniform}(0,360)$ degrees. Mutation size is sampled from the distribution $\sim \textrm{Gamma}(\alpha,\beta)$, where $\alpha$ and $\beta$ are chosen to give a mean mutation size of 0.6 units and a standard deviation of 0.4 units.  This distribution is parameterized so that mutation usually has little effect on antigenic phenotype, but occasionally has a large effect.  This is similar to the neutral networks implemented by Koelle et al. \cite{Koelle06}, wherein most amino acid changes result in little decrease to cross-immunity between strains, but some changes result in large jumps in cross-immunity.

Our model follows Gog and Grenfell \cite{Gog02} in representing antigenic distance as distance between points on a geometric space.  By forcing one-dimension to directly modulate $\beta$, Gog and Grenfell find an orderly replacement of strains.  Kryazhimskiy et al. \cite{Kryazhimskiy07} use a two-dimensional strain-space, but enforce a cross-immunity kernel that directly favors moving along a diagonal line away from previous strains.  Our model does not `build in' the one-dimensional direction of antigenic drift, which instead emerges dynamically from competition among strains.

\subsection*{Model output}

Daily incidence and prevalence are recorded for each region.  During the course of the simulation, samples of current infections are taken from the evolving virus population at a rate proportional to prevalence.  Each viral infection is assigned a unique ID, and in addition, infections have their phenotypes, locations and dates of infection recorded.  In this model, viruses lack sequences and so standard phylogenetic inference of the evolutionary relationships among strains is impossible.  Instead, the viral genealogy is directly recorded.  This is made possible by tracking transmission events connecting infections during the simulation; infections record the ID of their `parent' infection.  Proceeding from a sample of infections, their genealogical history can be reconstructed by following consecutive links to parental infections.  During this procedure, lineages coalesce to the ancestral lineages shared by the sampled infections, eventually arriving at the initial infection introduced at the beginning of the simulation.  Infections are sampled at a rate designed to give approximately 6000 samples over the course of the simulation, with genealogies reconstructed from a subsample of approximately 300 samples.

The results presented in figures \ref{incmaptree}--\ref{immunity} represents a single representative model output; one hundred replicate simulations were conducted to arrive at statistical estimates.  In 20 out of the 100 simulations, we observed a major bifurcation of antigenic phenotype and the consequent increase in incidence and genealogical diversity.  These simulations were removed from the analysis.   Similar to Koelle et al. \cite{Koelle11}, we assume that although the historical evolution of H3N2 influenza followed the path of a single lineage, it could have included a major bifurcation.  Further work in these directions will help to determine the likelihoods of single lineage vs.\ bifurcating scenarios.

\subsection*{Parameter selection and sensitivity analysis}

Estimating what the basic reproductive number $R_0$ for seasonal influenza would be in a naive population is notoriously difficult.  Season-to-season estimates of effective reproductive number $R$ for the USA and France gathered from mortality timeseries display an interquartile range of 0.9--1.8 \cite{Chowell08}.  Geographic spread within the USA suggests an average seasonal $R$ of 1.35 \cite{Viboud06}.  These estimates of $R$ will be lower than the $R_0$ of influenza due to the effects of human immunity.  We assumed $R_0$ of 1.8, consistent with the upper range of seasonal estimates.  Duration of infection was chosen based on patterns of viral shedding shown during challenge studies \cite{Carrat08}.  The linear form of the risk of infection and its increase as a function of antigenic distance $s$ was chosen as 0.07 based on experimental work on equine influenza \cite{Park09} and from studies of vaccine effectiveness \cite{Gupta06}.  Between-region contact rate $m$ was chosen to yield a trunk lineage that resides predominantly in the tropics.  With much higher rates of mixing, the trunk lineage ceases to show a preference the tropics, and with much lower rates of mixing, particular seasons in the north and the south will often be skipped.  The amplitiude of seasonal forcing $\epsilon$ was chosen to be just large enough to get consistent fade-outs in the summer months.

Mutational parameters were selected based on model output.  We chose $\mu$ by assuming 10 amino acid sites involved in antigenicity, each mutating at a rate of $10^{-5}$ \cite{Rambaut08} gives a phenotypic mutation rate $\mu = 10^{-4}$ per infection per day.  We chose mutational effect parameters that would give suitably fast rates of antigenic evolution corresponding to approximately 1.2 units of antigenic change per year, while simultaneously giving clustered patterns of antigenic evolution  \cite{Smith04}.  Similar outcomes are possible under a variety of parameterizations.  If mutations are more common ($\mu = 3 \times 10^{-4}$) and show less variation in effect size (mean = 0.6, sd = 0.2), then antigenic drift occurs in a more continuous fashion, resulting in less variation in seasonal incidence and a smoother distribution of antigenic phenotypes (Fig.~\ref{incmaptree_smooth}).  If mutations are less common ($\mu = 5 \times 10^{-5}$) and show more variance in effect (mean = 0.7, sd = 0.5), then antigenic change occurs in a more punctuated fashion (Fig.~\ref{incmaptree_rough}).  Basic reproductive number $R_0$ can be traded off with mutational parameters to some extent.  Less mutational input and higher $R_0$ will give similar patterns of antigenic drift and seasonal incidence.  Similarly, Kucharski and Gog \cite{Kucharski11} find that increasing $R_0$ results in increased rates of emergence of antigenically novel strains.

\subsection*{Antigenic map}

Antigenic phenotypes are modeled as discrete entities on the Euclidean plane; multiple samples have the same antigenic location.  However, in the empirical antigenic map of influenza A (H3N2), each strain appears in a unique location \cite{Smith04}.  We would argue that some of this pattern comes from experimental noise.  Indeed Smith et al. \cite{Smith04} find the distance between predicted measurements and observed measurements of antigen/antiserum pairs is on average 0.83 antigenic units with a standard deviation of 0.67 antigenic units.  We take this as a proxy for experimental noise and add jitter to each sampled antigenic phenotype by moving it in a random direction for an exponentially distributed distance with mean of 0.53 antigenic units.  If two samples with the same underlying antigenic phenotype are jittered in this fashion, the distance between them averages 0.83 antigenic units with a standard deviation of 0.64 units.

We added noise to each of the 5943 sampled viruses in this fashion resulting in an approximated antigenic map (Fig.~\ref{incmaptree}B).  Virus samples were then clustering following standard clustering algorithms.  We tried clustering by the $k$-means algorithm and also agglomerative hierarchical clustering with a variety of linkage criterion.  We found that clustering by Ward's criterion consistently outperformed other methods, when measured in terms of within-cluster and between-cluster variances.  However, the exact clustering algorithm had little effect on our overall results.

%%% REFERENCES %%%
\bibliographystyle{plos2009}
\bibliography{/Users/bedfordt/Documents/bedford}
