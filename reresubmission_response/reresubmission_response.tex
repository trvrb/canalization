\documentclass[stdletter,a4paper,noaddrfrom,orderfromdateto,11pt]{newlfm}
\noHeadline
\noFootline
\leftmarginsize{1.15in}
\rightmarginsize{1.15in}
\sigskipbefore{1cm}
\sigskipafter{0cm}
\headermarginskip{-0.3in}
\signature{Trevor Bedford}

\namefrom{}
\addrfrom{Inst.\ of Evolutionary Biology \\ University of Edinburgh}
\phonefrom{+44 774-259-5883}
\faxfrom{}
\emailfrom{t.bedford@ed.ac.uk}

\nameto{BMC Biology}
\addrto{BioMed Central \\ Floor 6, 236 Gray's Inn Road  \\ London, WC1X 8HB}

\greetto{Dear Dr.\ Jarvis,}
\closeline{Sincerely,}

\begin{document}
\begin{newlfm}

Thank you for your continued efforts in overseeing our manuscript.  We agree that discussion of the biological plausibility of our mutation parameters is very important in framing this work for a more general audience, especially with regards to the apparent parameter sensitivity to mutation rate shown in Figure 1.  Here, we tracked down estimates from the literature for the spontaneous rate of nucleotide mutation in influenza H3N2 and also for the number of amino acid sites most involved in determining antigenic phenotype.  We find that our estimate of $10^{-4}$ fits in nicely with estimates of the overall rate of antigenic mutation.  We discuss these findings in a new paragraph on page 8, copied below:

``For this process to take hold, the virus population needs to be somewhat mutationally-limited; if functional antigenic variants of novel phenotype emerge too quickly, then antigenic change will occur too rapidly for competition to winnow down the virus population to a single lineage (Figure~1).  Assuming that antigenic mutations have an average effect of 0.6 antigenic units and a standard deviation of 0.4 units, then the rate of new antigenic mutations cannot be greater than approximately $10^{-4}$ mutations per day (Figure~1).  Thus, it is important that the rate of $10^{-4}$ mutations per day be biological plausible.  Here, we take the rate of synonymous substitution as a proxy for the neutral rate of mutation.  The rate of synonymous change has been estimated at $2.5 \times 10^{-6}$ per site per day [24].  As there are approximately 2 nonsynonymous sites per codon in influenza [25], this gives a neutral rate of amino acid change of approximately $5 \times 10^{-6}$ per site per day.  Other work has shown that there appear to be $\sim$18 amino acid sites implicated in the majority of adaptive change [13].  These sites evolve along the trunk of the phylogeny at rate of 0.053 substitutions per site per year or at a combined rate of 0.95 substitutions per year [4].  This result agrees well with our finding of 0.81 antigenic mutations per year on the phylogeny trunk (Table~1).  If we assume 18 sites involved in antigenic change, this gives an overall rate of antigenic mutation of $9 \times 10^{-5}$ per day.  Thus, we believe that $10^{-4}$ mutations per day represents a biologically reasonable estimate.''

Although we hope that this fully clarifies the biological underpinnings for our choice of mutation rate, we would be happy to work on this issue further if the editorial board thinks it is necessary.

\end{newlfm}
\end{document}  